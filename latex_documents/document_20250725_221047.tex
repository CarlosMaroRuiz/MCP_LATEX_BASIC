\documentclass[12pt,a4paper]{article}
\usepackage[utf8]{inputenc}
\usepackage[T1]{fontenc}
\usepackage[spanish]{babel}
\usepackage{amsmath,amssymb,amsthm}
\usepackage{graphicx}
\usepackage[colorlinks=true,linkcolor=blue,urlcolor=blue,citecolor=red]{hyperref}
\usepackage{booktabs}
\usepackage{array}
\usepackage{float}
\usepackage{enumitem}
\usepackage{fancyhdr}
\usepackage[margin=2.5cm]{geometry}
\usepackage{titlesec}
\usepackage{xcolor}
\usepackage{setspace}
\usepackage{lipsum}

\title{Dinámica de los Ecosistemas de Bosque Tropical: Interacciones y Conservación}
\author{Nombre del Investigador}
\date{\today}

\onehalfspacing
\setlength{\parindent}{1.25cm}
\titleformat{\section}{\large\bfseries}{\thesection.}{0.5em}{}
\titleformat{\subsection}{\normalsize\bfseries}{\thesubsection.}{0.5em}{}

\begin{document}

\maketitle

\begin{abstract}
Este estudio examina la compleja dinámica de los ecosistemas de bosque tropical, analizando interacciones bióticas, impactos del cambio climático y estrategias de conservación. Mediante revisión bibliográfica sistemática y análisis de datos ecológicos, se identifican patrones clave de biodiversidad y vulnerabilidad. Los resultados destacan la importancia de enfoques integrados para la sostenibilidad, considerando tanto factores ecológicos como socioeconómicos en modelos de conservación adaptativa.
\end{abstract}

\section{Introducción a los Ecosistemas de Bosque Tropical}
Los bosques tropicales cubren aproximadamente el 7\% de la superficie terrestre pero albergan más del 50\% de la biodiversidad global \cite{gaston2000}. Estos ecosistemas se caracterizan por su estructura estratificada y complejas redes tróficas:

\begin{equation}
H' = -\sum_{i=1}^{S} p_i \ln p_i
\end{equation}
donde \(H'\) es el índice de Shannon-Wiener y \(p_i\) la proporción de individuos de la especie \(i\). La alta productividad primaria (\(>2000\) g C/m\textsuperscript{2}/año) sustenta esta diversidad \cite{malhi2011}.

\begin{figure}[H]
\centering
\includegraphics[width=0.8\textwidth]{estructura_bosque}
\caption{Diagrama de estratificación típica en bosques tropicales (dosel, sotobosque, suelo)}
\label{fig:estratificacion}
\end{figure}

\section{Biodiversidad y Complejidad Ecológica}
La coexistencia de especies se explica mediante modelos de nicho ecológico:
\begin{align*}
\frac{dN_i}{dt} &= r_i N_i \left(1 - \frac{\sum_{j=1}^{n} \alpha_{ij} N_j}{K_i}\right) \\
\alpha_{ij} &: \text{coeficiente de competencia interespecífica}
\end{align*}

\begin{table}[H]
\centering
\caption{Indicadores de biodiversidad en bosques tropicales seleccionados}
\label{tab:biodiversidad}
\begin{tabular}{@{}lccc@{}}
\toprule
\textbf{Región} & \textbf{Especies/ha} & \textbf{Endemismo (\%)} & \textbf{Índice Simpson} \\
\midrule
Amazonía & 285 & 32 & 0.92 \\
Congo & 201 & 28 & 0.87 \\
Borneo & 240 & 41 & 0.94 \\
\bottomrule
\end{tabular}
\end{table}

\section{Impacto del Cambio Climático}
El incremento de \(CO_2\) (\(>415\) ppm) altera los ciclos biogeoquímicos:
\[\Delta GPP = \beta \cdot \ln\left(\frac{[CO_2]_a}{[CO_2]_b}\right) + \varepsilon\]
donde \(\beta = 0.65\) para bosques tropicales \cite{hubau2020}. La fragmentación aumenta la vulnerabilidad:

\begin{itemize}[label=\textbullet]
\item Reducción del 35\% en conectividad de corredores (2000-2020)
\item Aumento de \(2.5^\circ C\) en bordes de fragmentos
\item Disminución del 40\% en biomasa arbórea en fragmentos <10 ha
\end{itemize}

\section{Estrategias de Conservación}
Enfoques efectivos incluyen:
\begin{enumerate}[label=(\arabic*)]
\item Corredores biológicos integrados
\item Pagos por servicios ecosistémicos (PSE)
\item Manejo comunitario con gobernanza local
\end{enumerate}

La efectividad de las Áreas Protegidas (APs) sigue:
\[E_{AP} = \frac{A_{intacta}}{A_{total}} \times \gamma_{gestión} \times C_{conectividad}\]

\section{Conclusiones}
Los bosques tropicales requieren estrategias policéntricas que integren:
\begin{itemize}[label=\textbullet]
\item Modelos predictivos de resiliencia climática
\item Políticas transnacionales coordinadas
\item Indicadores de sostenibilidad multidimensionales
\end{itemize}
La ventana de acción se estima en 10-15 años para puntos de no retorno \cite{lovejoy2019}.

\begin{thebibliography}{99}
\bibitem{gaston2000} Gaston, K.J. (2000). \textit{Global patterns in biodiversity}. Nature, 405:220-227.

\bibitem{malhi2011} Malhi, Y. (2011). \textit{The productivity, metabolism and carbon cycle of tropical forest vegetation}. Journal of Ecology, 99(3):650-657.

\bibitem{hubau2020} Hubau, W. et al. (2020). \textit{Asynchronous carbon sink saturation in African and Amazonian tropical forests}. Nature, 579:80-87.

\bibitem{lovejoy2019} Lovejoy, T.E., Nobre, C. (2019). \textit{Amazon tipping point}. Science Advances, 4(2):eaat2340.

\bibitem{web1} Global Forest Watch (2023). \textit{Interactive forest monitoring}. Disponible en: \url{https://www.globalforestwatch.org}
\end{thebibliography}

\end{document}
