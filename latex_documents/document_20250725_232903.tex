\documentclass[12pt,a4paper]{article}
\usepackage[utf8]{inputenc}
\usepackage[T1]{fontenc}
\usepackage[spanish]{babel}
\usepackage{amsmath,amssymb,amsthm}
\usepackage{graphicx}
\usepackage[colorlinks=true,linkcolor=blue,urlcolor=blue,citecolor=red]{hyperref}
\usepackage{booktabs}
\usepackage{array}
\usepackage{float}
\usepackage{enumitem}
\usepackage{fancyhdr}
\usepackage[margin=2.5cm]{geometry}
\usepackage{titlesec}
\usepackage{xcolor}
\usepackage{setspace}
\usepackage{lipsum}

\title{Análisis de Sistemas Distribuidos en Entornos Cloud}
\author{María Rodríguez Fernández}
\date{15 de octubre de 2023}

\pagestyle{fancy}
\fancyhf{}
\rhead{Análisis de Sistemas Distribuidos}
\lhead{\thepage}
\renewcommand{\headrulewidth}{0.4pt}

\begin{document}

\maketitle

\begin{abstract}
\noindent Este estudio examina los desafíos y soluciones en sistemas distribuidos modernos implementados en entornos cloud. Se analizan patrones arquitectónicos, mecanismos de consistencia de datos y estrategias de tolerancia a fallos mediante experimentos controlados en tres plataformas cloud diferentes. Los resultados demuestran una mejora del 47\% en rendimiento utilizando el enfoque propuesto de coordinación descentralizada, comparado con sistemas tradicionales basados en líder. Se identifican además puntos críticos de optimización para cargas de trabajo heterogéneas.
\end{abstract}

\section{Introducción}
Los sistemas distribuidos constituyen la columna vertebral de la infraestructura cloud moderna, soportando aplicaciones desde comercio electrónico hasta procesamiento científico de datos. Su complejidad inherente deriva de múltiples factores concurrentes: consistencia de datos, tolerancia a fallos, coordinación de nodos y equilibrio de carga \cite{tanenbaum2017}.

\subsection{Contexto tecnológico}
La evolución hacia arquitecturas microservicios ha incrementado exponencialmente los requisitos de comunicación inter-servicio. Según estudios recientes \cite{adya2019}, más del 65\% de la latencia en aplicaciones distribuidas modernas se atribuye a sobrecarga de coordinación. Esto plantea desafíos fundamentales:

\begin{equation}
\mathcal{L}_{total} = \sum_{i=1}^{n} \left( \mathcal{L}_{comp}^i + \mathcal{L}_{coord}^i \right) + \max_{j} \left( \mathcal{L}_{net}^j \right)
\label{eq:latencia}
\end{equation}

donde $\mathcal{L}_{comp}$ representa latencia computacional, $\mathcal{L}_{coord}$ latencia de coordinación, y $\mathcal{L}_{net}$ latencia de red.

\subsubsection{Problemática central}
La tensión fundamental en diseño de sistemas distribuidos se manifiesta en el triángulo CAP \cite{brewer2012}:
\begin{itemize}
\item \textbf{Consistencia}: Uniformidad de datos en todos los nodos
\item \textbf{Disponibilidad}: Respuesta ante todas las solicitudes
\item \textbf{Tolerancia a particiones}: Operación durante fallos de red
\end{itemize}

\begin{table}[H]
\centering
\caption{Comparativa de enfoques de consistencia}
\label{tab:consistencia}
\begin{tabular}{@{}lccc@{}}
\toprule
\textbf{Modelo} & \textbf{Latencia (ms)} & \textbf{Tolerancia fallos} & \textbf{Complejidad} \\
\midrule
Strong Consistency & 152 & Baja & Alta \\
Eventual Consistency & 47 & Media & Media \\
CRDTs & 63 & Alta & Alta \\
\bottomrule
\end{tabular}
\end{table}

Como muestra la Tabla \ref{tab:consistencia}, no existe solución óptima universal. Nuestra investigación se centra en estrategias adaptativas que modifiquen dinámicamente el modelo de consistencia según patrones de carga observados.

\subsection{Objetivos}
Este trabajo busca:
\begin{enumerate}[label=(\roman*)]
\item Cuantificar el impacto de protocolos de consenso en rendimiento
\item Diseñar mecanismos de transición suave entre modelos de consistencia
\item Evaluar estrategias de recuperación ante fallos catastróficos
\end{enumerate}

\lipsum[1-3]

\section{Conclusiones}
Los experimentos realizados demuestran la viabilidad del modelo de consistencia adaptativa propuesto. Mediante monitorización en tiempo real de 15 métricas clave (Figura \ref{fig:metricas}), el sistema ajusta automáticamente su comportamiento:

\begin{figure}[H]
\centering
\includegraphics[width=0.7\textwidth]{metricas_rendimiento}
\caption{Métricas de rendimiento en diferentes regímenes de carga}
\label{fig:metricas}
\end{figure}

\subsection{Hallazgos principales}
Los resultados obtenidos permiten afirmar:
\begin{itemize}
\item La estrategia CRDT+ mostró un 23\% menos de latencia que CRDTs convencionales
\item El protocolo de consenso basado en RAFT generó cuellos de botella en cargas asimétricas
\item El modelo predictivo redujo fallos de coordinación en un 81\% durante particiones de red
\end{itemize}

\subsubsection{Lecciones aprendidas}
La principal contribución de esta investigación radica en la implementación práctica de:
\begin{equation}
\mathcal{C}_{adapt} = \alpha \cdot \mathcal{M}_{lat} + \beta \cdot \mathcal{M}_{disp} + \gamma \cdot \mathcal{M}_{cons}
\label{eq:modelo}
\end{equation}

donde los coeficientes $\alpha$, $\beta$ y $\gamma$ se ajustan dinámicamente según condiciones operativas.

\subsection{Trabajo futuro}
Quedan abiertas varias líneas de investigación:
\begin{enumerate}
\item Integración con sistemas serverless (FaaS)
\item Optimización para hardware heterogéneo (GPU/TPU)
\item Extensiones para entornos edge computing
\end{enumerate}

La implementación de referencia está disponible en: \url{https://github.com/dist-sys/adaptive-crdt}

\begin{thebibliography}{9}
\bibitem{tanenbaum2017} Tanenbaum, A., \& Van Steen, M. (2017). \textit{Distributed Systems}. Pearson.
\bibitem{adya2019} Adya, A. et al. (2019). Slicer: Auto-Sharding for Datacenter Applications. \textit{OSDI}.
\bibitem{brewer2012} Brewer, E. (2012). CAP Twelve Years Later: How the "Rules" Have Changed. \textit{Computer}, 45(2).
\end{thebibliography}

\end{document}
