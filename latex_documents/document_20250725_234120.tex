\documentclass[12pt,a4paper]{article}
\usepackage[utf8]{inputenc}
\usepackage[T1]{fontenc}
\usepackage[spanish]{babel}
\usepackage{amsmath,amssymb,amsthm}
\usepackage{graphicx}
\usepackage[colorlinks=true,linkcolor=blue,urlcolor=blue,citecolor=red]{hyperref}
\usepackage{booktabs}
\usepackage{array}
\usepackage{float}
\usepackage{enumitem}
\usepackage{fancyhdr}
\usepackage[margin=2.5cm]{geometry}
\usepackage{titlesec}
\usepackage{xcolor}
\usepackage{setspace}
\usepackage{lipsum}

\title{Informe Técnico: Inteligencia Artificial}
\author{Departamento de Ciencias de la Computación}
\date{\today}

\setlength{\parindent}{0pt}
\setlength{\parskip}{1em}
\titlespacing*{\section}{0pt}{1.5em}{0.8em}
\titleformat{\section}{\large\bfseries}{\thesection}{1em}{}

\begin{document}

\begin{titlepage}
\centering
\vspace*{2cm}
{\LARGE \textbf{HOJA INFORMATIVA SOBRE INTELIGENCIA ARTIFICIAL}}\\[1cm]
{\Large Fundamentos, aplicaciones y consideraciones éticas}\\[2cm]
\includegraphics[width=0.4\textwidth]{example-image}\\[1cm]
{\large Elaborado por:}\\[0.5cm]
{\Large \textbf{Departamento de Investigación en IA}}\\[2cm]
{\large Instituto de Tecnología Avanzada}\\[0.5cm]
{\large Ciudad del Conocimiento}\\[2cm]
{\large \today}
\end{titlepage}

\tableofcontents
\newpage

\section*{Resumen Ejecutivo}
\addcontentsline{toc}{section}{Resumen Ejecutivo}
La Inteligencia Artificial (IA) transforma radicalmente sectores económicos y sociales mediante sistemas capaces de realizar tareas que requieren inteligencia humana. Este documento analiza su definición, evolución histórica, aplicaciones contemporáneas y desafíos éticos emergentes, proporcionando una visión integral de su impacto actual y futuro.

\section{Definición de Inteligencia Artificial}
La Inteligencia Artificial se define como la disciplina científica que desarrolla sistemas capaces de realizar tareas que normalmente requieren inteligencia humana. Estos sistemas exhiben características como:

\begin{itemize}
\item \textbf{Aprendizaje automático:} Capacidad de mejorar el rendimiento mediante experiencia
\item \textbf{Razonamiento lógico:} Habilidad para resolver problemas y tomar decisiones
\item \textbf{Percepción:} Interpretación de datos sensoriales (visuales, auditivos, etc.)
\item \textbf{Interacción natural:} Comunicación en lenguaje humano mediante NLP
\end{itemize}

Matemáticamente, un sistema de IA puede modelarse como:
\begin{equation}
f: X \rightarrow Y
\end{equation}
donde \(X\) representa los datos de entrada e \(Y\) las acciones o predicciones, optimizadas mediante:
\begin{equation}
\arg\min_{\theta} \mathcal{L}(f_\theta(x), y)
\end{equation}
siendo \(\mathcal{L}\) la función de pérdida y \(\theta\) los parámetros del modelo.

\section{Historia de la IA}
La evolución de la IA se divide en cuatro etapas clave:

\begin{table}[H]
\centering
\caption{Hitos históricos de la IA}
\label{tab:historia}
\begin{tabular}{@{}lll@{}}
\toprule
\textbf{Periodo} & \textbf{Evento} & \textbf{Significado} \\
\midrule
1950-1956 & Conferencia de Dartmouth & Nacimiento formal de la disciplina \\
1956-1974 & Sistemas expertos & Primeras aplicaciones prácticas \\
1980-1990 & Invierno de la IA & Reducción de financiamiento \\
2010-actual & Deep Learning & Avances revolucionarios en reconocimiento \\
\bottomrule
\end{tabular}
\end{table}

\section{Aplicaciones Actuales}
Principales implementaciones contemporáneas:

\subsection*{Sectores Transformados}
\begin{itemize}
\item \textbf{Salud:} Diagnóstico médico por imagen (precisión del 92\% en detección de tumores)
\item \textbf{Finanzas:} Sistemas antifraude que analizan 10,000 transacciones/segundo
\item \textbf{Manufactura:} Mantenimiento predictivo que reduce fallos en 45\%
\item \textbf{Transporte:} Vehículos autónomos con más de 300 millones de km recorridos
\end{itemize}

\subsection*{Tecnologías Clave}
\begin{enumerate}[label=\textbf{\arabic*.}]
\item Procesamiento de Lenguaje Natural (ChatGPT, asistentes virtuales)
\item Visión por computador (reconocimiento facial, análisis de imágenes médicas)
\item Sistemas de recomendación (Amazon, Netflix, Spotify)
\item Robótica autónoma (logística, exploración espacial)
\end{enumerate}

\section{Desafíos Éticos}
Consideraciones críticas en el desarrollo responsable:

\begin{itemize}
\item \textbf{Sesgos algorítmicos:} Discriminación en sistemas de contratación (\href{https://www.weforum.org}{WEF Report 2023})
\item \textbf{Privacidad:} Uso de datos personales sin consentimiento explícito
\item \textbf{Transparencia:} Opacidad en decisiones de modelos complejos (\"caja negra\")
\item \textbf{Impacto laboral:} Automatización del 47\% de empleos para 2035 (Estudio McKinsey)
\item \textbf{Control autónomo:} Responsabilidad en sistemas de armamento letal
\end{itemize}

\section*{Conclusiones}
\addcontentsline{toc}{section}{Conclusiones}
La IA representa una tecnología transformadora con potencial para incrementar la productividad global en un 40\% según el \href{https://www.pwc.com}{PwC AI Report}. Su desarrollo debe equilibrar innovación con marcos éticos robustos, especialmente en áreas sensibles como salud y justicia. La colaboración multidisciplinaria resulta esencial para maximizar beneficios sociales y mitigar riesgos.

\begin{thebibliography}{9}
\bibitem{russell} Russell, S., \& Norvig, P. (2020). \textit{Artificial Intelligence: A Modern Approach}. Pearson.
\bibitem{floridi} Floridi, L. (2019). \textit{Ética de la Inteligencia Artificial}. Oxford University Press.
\bibitem{aiindex} Stanford HAI (2023). \textit{AI Index Report}. Disponible en: \url{https://aiindex.stanford.edu}
\end{thebibliography}

\end{document}
